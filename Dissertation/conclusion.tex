\chapter{Conclusion}
In this chapter, the author will discuss the conclusions drawn throughout the projects development and how or if the initial goals laid out in the introduction of this paper were met.

\section{Overview}
In the introduction of this paper the objectives of the projects were specified as: \\ \\

\begin{itemize}
    \item Produce a simple, easy to use and understand, web application
    \item Gain further understanding of web technologies
    \item Create an e-commerce web application that meets modern expectations
    \item Complete the given assignment as a lone venture
    \item Introduce the concept of the project
    \item Provide an understanding of web technologies
    \item Outline the development of the applied project
\end{itemize}

\newpage

It can be determined that the main goal of this project had been for the web-application that was produced to have had a simple nature and to be easy to use and understand. Given the feedback the author received from the acceptance testing section of the system evaluation chapter it is of the authors opinion to conclude this goal as a success. \\ \\
The author concludes that the goal of gaining a further understanding of web technologies has also been achieved as the author has developed a functioning application which utilizes technologies such as Firebase which had not been used by the author previously and the work completed within the applied project is proof of furthering their knowledge in using technologies such as Angular and Node with which they had previous experience with. \\ \\
The goal of making an e-commerce web-applications was met for the most part, although feedback provided from fellow students was positive for the most part it was almost consensus among those that partook in the Usability and Acceptance testing stages that an e-commerce application such as this should be able to provide the functionality to process a users order. \\ \\
In the opinion of the author, completing the assignment as a solo venture was also a success. Although the author was originally to undertake this project as a team exercise, the realization that the scope of the project was not grand enough to be completed by a 3-person party led to the original team to be dissolved. However, the research conducted during this time was still applied to this project as the scope of the project and its them stayed largely intact. Completing the project and meeting the user requirements outlined in the introduction to this paper, of what was originally intended to be approached by a group rather than a single person, lead to the conclusion that this goal was a success. \\ \\
The author feels that throughout the introduction of the project that the concept of the project was correctly introduced to potential readers and is presented in a way that is understandable to those without heavy experience in the area explored. \\ \\ 
The author can conclude that readers where suitably provided with an understanding of the web technologies implemented in the development of this project. This is concluded from the third chapter of this paper wherein the author extensively explains which web technologies are used and how they are used with the application. \\ \\
The author feels that the development of the web-application was outlined in a fine manner as the chapter on the System design highlights the construction of all the components within the web application and discuss the back end technologies employed too.

\section{Future Modifications}
It is the opinion of the author that the application has numerous routes in which it can grow. Firebase authentication allows for user authentication to be provided from many sources, at the moment the application allows for application to be accessed for users with Google accounts and for classic email and password inputs, however Firebase allows for authentication through almost any social platform, by phone or even by Github account. \\ \\
Another future modification is the use of higher tiers of data storage from Firebases' servers. Currently the application only employs the use of the free tier of Firebases' database service, for a fee this can be upgraded to faster, more responsive servers. \\ \\
Aside from this, some obvious modification can be made to rectify the lack of order placement functionality and the faulty search functionality within the application. It can also be seen that a more thorough, professional styling can be applied throughout the application and that it can be geared to words looking presentable on mobile devices as was outlined in the feedback of some of the testing. 

\section{Final Conclusion}
Overall the author feels like the experience of the development of the application was a strong learning experience. The weekly meetings with the project supervisor provided a pseudo-customer experience to the assignment not present in other modules on the course. The author also feels that the module allowed for a significant amount of self learning, which is something the course has always pushed as something the participants should strive for.

\section{Final Words}
As the author prepares to submit this document and bring a close to the majority of work in their final year of education at GMIT, it is a pleasant experience to be able to compare the application delivered as part of the project compared to the similar project they were tasked with in their first year of education at GMIT and be able to see solid proof of their growth as a software developer.

\chapter{Appendix}

Link to Github: https://github.com/CookeRichard94/fyp \\
Link to Deployed application: https://fyp-server-75156.web.app/ 



















