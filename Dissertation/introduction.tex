
{\noindent\Large\textbf {Abstract:}} \\  \\ E-commerce continues to play an ever expanding role in modern society, multinational corporations and the information technology sector. E-commerce is now the leading way in which business to consumer transactions occur. This document details the creation of a modern-day e-commerce web application. The document details how this was planned, solved and the methods that were used in order to meet the requirements of a modern day customer. The solution to this was to create a full-stack web application that allows users to login or register to the application, view products, add them to cart, review them, order them and personalize user details. Restrictions on the creation and deletion of products are limited to those with administrative rights. \\ 

{\noindent\Large\textbf {Author:}} \\  This project was developed as part of the 15 credit module, Applied Project and Minor Dissertation, by Richard Cooke, a final year student of the Computing and Software Development course at Galway-Mayo Institute of Technology(GMIT).\\

{\noindent\Large\textbf {Acknowledgements:}} \\ The author would like to thank project supervisor Mr. Gerard Harrison and module supervisor Dr. John Healy for their time and support throughout the project development. 


\chapter{Introduction}
This chapter serves as an introduction to the project. In it, the various aspects
of the project are discussed and the main objectives of the project are outlined. Each of the 
chapters are briefly discussed as well, detailing what each contains and what is
discussed within each. \\ \\

Throughout the first three years of study at GMIT, students, the author included, were educated on a broad range of technologies in both hardware and software. This was conceived so that graduating students would be better prepared for a wide range of challenges in the post-graduate workplace.\\
At the commencement of our 4th year, we were informed that a year-long, 15 credit project would have to submitted by the end of the second semester of the current academic year. Following this it was determined that I would perform this task as part of a 3-person group, however this group would later split with both parties keeping the original idea in slightly different formats. \\
With input from the project supervisor, the author began to research which technologies could form the basis of this project. Discussions with the project supervisor occurred on a weekly basis to ascertain that the project was remaining within the scope of what would be expected of a level 8 project. Due to the overwhelming presence of e-commerce in modern day life the author found this to be the most suitable area to explore project wise, the author also determined this would be an accurate way to gauge the growth in their software development skills as one of the end of year project for delivered in the first year of study at GMIT had a similar basis. 

\newpage
\section{Project Objectives} 
The main goal of this project is to produce a web application that mimics a users experience on e-commerce sites such as Amazon. Along with this the author would like to gain a better understanding of e-commerce, the web-technologies involved in e-commerce development and to attempt to gauge the developers own development in working with web technologies. \\
The project can be divided into 2 segments, the applied project itself and this  dissertation. The dissertation will be used as a means for containing the research conducted in the projects development whereas the applied project will outline the technologies involved in the projects development. \\ \\
The objectives of the applied project, as determined by the author, where as follows : \\ \\
\textbf{\emph{Produce a simple, easy to use and understand, web application.}} This application will make use of a number of complex algorithms with a sophisticated back end server and database. The complexities of these features are to remain hidden from users and users must be presented with an easily understandable and navigable front-end, which should not require the user to have anything above base understandings of web applications. \\ \\
\textbf{\emph{Gain further understanding of web technologies.}} The development of this project will be used as a time for the author to both learn and work with new web technologies and for the author to further develop their understanding of other, previously used, web technologies. This will be done to try and maximize the learning outcomes of the module. \\  \\
\textbf{\emph{Create an e-commerce web application that meets modern expectations.}} The end goal of the applied project is to have produced a web application that has an environment similar to that of modern, multinational e-commerce web applications such as Amazon or eBay. \\  \\
\textbf{\emph{Complete the given assignment as a lone venture.}} This project will be undertaken as solo endeavour. This is to be achieved by adhering to industry standards and methodologies, which has been done to simulate the expected experience of working on a similar scenario in a real workplace scenario. As the entirety of development responsibilities lay solely with a singular developer it is imperative that the project is approached so that maximum efficiency is aimed for. \\ \\
\newpage

The objectives of the dissertation, as determined by the author, where as follows: \\ \\
\textbf{\emph{Introduce the concept of the project.}} The author of this paper will aim to provide the user with an introduction to the project that accurately describes its inspirations and end-goals. \\ \\
\textbf{\emph{Provide an understanding of web technologies.}} The number of web technologies available for web applications continues to grow at a near exponential rate. Within this paper the author will discuss a range of these technologies, The author will also argue why the technologies that were used were the most appropriate for the projects development. \\ \\
\textbf{\emph{Outline the development of the applied project.}} One of the main aims of the paper is to give its readers a complete breakdown of the projects development cycle from the initial conception, to the research involved and on to the eventual completion of the applied project. The methodologies and technologies used in the projects development, along with an evaluation of the system as well as a discussion of the systems design. Following this, the issues that the were encountered by the author throughout the development of the project in both the applied project and in other scenarios, how these problems were solved and what would have been done differently if the project were restarted. To conclude this paper the author will evaluate the project as a whole. \\ \\

\section{Metrics for Success and Failure}
In an attempt to control and more easily manage the development of the project, the author outlined the metrics that would mean either success or failure for the project, in its early stages. Doing this allowed for the author to track their progress and to ascertain that the project maintained its course in achieving the project objectives that were earlier outlined. \\ \\
The metrics for which the success of the applied project and the dissertation are as follows: \\ \\
\textbf{\emph{A comprehensive dissertation that can be understood by anyone,  regardless of their initial knowledge level of the technologies implemented in the applied project.}} This was achieved by making conceptual drafts for each section that was to be included in the finalized version of the dissertation and these early drafts were shared with and read by students who were attempting similar projects, this service was also done for these students who wanted their drafts read in an attempt to cultivate better understanding of what expectations were for what was being demanded.\\ \\
\textbf{\emph{An easily used web application, that fully functions.}} This was achieved using similar methodology as with the dissertation. As the application was being developed, beta versions where distributed to fellow students of the author so as to ascertain that the application was meeting the preassigned objectives of being easily used and accessed and maintaining full functionality. 

\section{Chapter Outline} 
This paper has been organized into a standard layout of chapters, with the purpose of each of these chapters being to explore the various, different, aspects of the projects. 
These chapters are outlined as follows: 
\subsection{Methodology}
This chapter will outline the approaches that the author used in the planning, organization, management and development of the project. The section aims to enlighten the reader on how the projects progressed from research to the final applied project. The author will discuss the what methodologies where implemented in the development of the project and why these methodologies were chosen. 
\subsection{Technology Review} 
This chapter will outline the technologies that were used in the final version of the project. The technologies will be explained by the author and their implementation in the project will be outlined along with why they were chosen as the technologies to be used in the projects development over other, similar, technologies.
\subsection{System Design} 
This chapter will outline the architecture and design of the web applications
system. Using code snippets and visual diagrams to help the reader attain a basic
understanding of the web application design. This chapter will provided an in-depth exploration of all the layers of the system, the data, logic and presentation layers. 
\subsection{System Evaluation}
This chapter will provide an evaluation of the software developed for the final version of the applied project. The author will evaluate areas of the system such as its robustness, scalability and testing of the system. The results of this evaluation will be used as a measurement to compare with the objectives outlined in the introduction section. The author will also discuss what opportunities could have been taken to improve the system and software developed. 
\subsection{Conclusion} 
This chapter will provide a review of the goals outlined at the onset of the project. It will highlight any findings from the System Evaluation chapter, giving a final analysis on what discoveries were made through the development of the project, before the author starts a brief discussion on their experience of the project. 

\section{User Requirements} \\ \\
\begin{itemize}
	\item User must be able to register an account
	\item User must be able to log in with email and password
	\item User must be able to log in with Google account
	\item Log in must be persistent throughout a session
	\item User must be able to log out
	\item User must be able to add personal details
	\item User must be able to view product
	\item User must be able to add product to cart
	\item User must be able to view cart
	\item User must be able to remove cart item
	\item User must be able to review product
	\item User must not be able to access "Admin" section
	\item Admin must be able to view list of user
	\item Admin must be able to add products
	\item Admin must be able to edit products
	\item Admin must be able to delete products
	\item Admin must be able to attach picture to product
\end{itemize} \\ \\

\section{Github} 
In the case of this project was used for version control. The git repository also contains a readme which contains instructions on how to install and run the project on a local machine. \\
The Github repository can be found at: \\
https://github.com/CookeRichard94/fyp
